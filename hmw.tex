\documentclass[a4paper,12pt]{article}
\usepackage[utf8x]{inputenc}
\usepackage[english,russian]{babel}
\usepackage{cmap}

\title{Суммы биномиальных коэффициентов}
\author{Рахманбердиев C.}
\date{13433/3}

\begin{document}
	\maketitle
\newpage
Установим несколько оценок для сумм биномиальных коэффициентов. Первая оценка доказывается так же, как и известное в теории вероятностей
неравенство Чебышева.


\textbf{Теорема 2.8}
Пусть 1 ≤ $\phi(n) ≤ \frac{\sqrt{n}}{2}$.Тогда
$$\sum_{k=0}^{n/2-\sqrt{n\phi (n)}} (\frac{n}{k}) ≤ \frac{2^{n-3}}{\phi (n)}$$
\textbf{Доказательство.} Оценим сумму биномиальных коэффициентов, ниж-
ний индекс которых отличается от n/2
 более чем на t единиц:
 \begin{equation}
 \sum_{k:n/2 - k >t} (\frac{n}{k}) = \sum_{k:n/2-k>t} \frac{(n/2 -k)^2}{(n/2 -k)^2} (\frac{n}{k}) ≤
 \frac{1}{t^2} \sum_{k:n/2-k>t} (n/2 -k )^2  (\frac{n}{k})  ≤ \frac{1}{t^2} \sum_{k=0}^{n}(n/2 -k )^2  (\frac{n}{k})
 \end{equation}
\\ Найдем сумму, стоящую в правой части неравенства (1). Легко видеть,
 что
 $$\sum_{k=0}^{n}(n/2 -k )^2  (\frac{n}{k})=\sum_{k=0}^{n}(\frac{n}{k})(n^2/4-nk+k^2)=n^2/4\sum_{k=0}^{n}(\frac{n}{k})-\sum_{k=0}^{n}(\frac{n}{k})(n-k)k$$
 Из двух предыдущих неравенств следует, что
$$\sum_{k=0}^{n}(\frac{n}{k})(n/2 -k )^2=n^2 2^{n-2}-n(n-1)2^{n-2}=n2^{n-2}$$
Подставляя полученное равенство в правую часть (1) и полагая t равным
$\sqrt{n\phi (n)}$ находим, что
$$\sum_{k:n/2-k} (\frac{n}{k})<\frac{n2^{n-2}}{n\phi (n)}=\frac{2^{n-2}}{\phi (n)}$$
\\
Теорема доказана.\\
Неравенство теоремы 2.8 достаточно грубое (далее оно будет существен-
но усилено в теореме 2.10). Тем не менее метод, которым оно получено,
представляет значительный интерес и может быть успешно использован в
различных задачах.\\
\\
\textbf{Теорема 2.10.}\\ При 0 ≤ t ≤ n/2 справедливо неравенство
$$\sum_{k=0}^{n/2-t}(\frac{n}{k})≤2^ne^{\frac{-2t^2}{n}}$$
\textbf{Доказательство}. Из теоремы 2.8 и доказанного на стр. 2 равенства
(1) следует, что
\begin{equation}
2^{nH(1-2t/2)}≤2^{n(1-1/2((1-2t/n)log_2(1-2t/n)+(1+2t/n)log_2(1+2t/n)}
\end{equation}
Для того, чтобы оценить показатель экспоненты в правой части нера-
венства (2) покажем, что
$$f(x)=(1-x)ln(1-x)+(1+x)ln(1+x)-x^2>0$$
\\при
$x
∈ (−1, 1).$ Прежде всего заметим, что
$f(x)$ — четная функция. Следовательно, справедливость доказываемого неравенства достаточно установить только для полуинтервала [0, 1), а так как
$f(0)$ = 0, то достаточно
показать, что на этом полуинтервале производная функции
$f(x)$ неотрица
тельна. Дифференцируя
$f(x)$, находим
$$f(x)=ln(1+x)-ln(1-x)-2x$$
Нетрудно видеть, что
$f(0)$ = 0 и вторая производная
$$f(x)=2/(1-x^2)-2$$
функции
f(x) на [0, 1) неотрицательна. Таким образом,
f
(x) ≥ 0 на [0, 1),
и поэтому,
$$(1-x)ln(1-x)+(1+x)ln(1+x)>x^2$$
\\при всех
x из интервала (−1, 1). Следовательно,
\begin{equation}
(1-2t/n)log_2(1-2t/n)-(1+2t/n)log_2(1+2t/n)<rt^2/n^2log_2e
\end{equation}
\\Подставляя неравенство (3) в неравенство (2), нетрудно видеть,
что
$$\sum_{k=0}^{n/2-t} \frac{n}{k}<2^n(1-1/2\cdot4t^2/n^2log_2e)=2^ne^{-2t^2/n}$$
\\Теорема доказана.
\newpage
Содержание
\section{Суммы биномиальных коэффициентов}
\subsection{Теорема 2.8\ldots\ldots\ldots\ldots\ldots\ldots\ldots\ldots\ldots\ldots\ldots\ldots\ldots2}
\subsection{Теорема 2.9\ldots\ldots\ldots\ldots\ldots\ldots\ldots\ldots\ldots\ldots\ldots\ldots\ldots2}
\subsection{Теорема 2.10\ldots\ldots\ldots\ldots\ldots\ldots\ldots\ldots\ldots\ldots\ldots\ldots\ldots3}
\end{document}